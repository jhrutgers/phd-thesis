% vim:spelllang=nl
\chapter*{Dankwoord}
\pdfbookmark{Dankwoord}{acknowledgments}

\def\ppp{ \ldots\xspace}
\def\stick#1#2#3{\raisebox{#1}[0pt][0pt]{\rotatebox{#2}{#3}}\hspace{-.25ex}}

E\'en woord per acht minuten.
Als je dat gedurende het promotietraject toevoegt aan je proefschrift, dan heb je aan het eind voldoende omvangrijk werk geleverd.
Gegeven een gemiddelde typesnelheid van 50~woorden per minuut, ben je dus \SI[output-decimal-marker={,}]{.25}{\percent} van je tijd bezig met je proefschrift.
Dat geeft lekker veel tijd voor het onderzoek zelf.

Ik weet nog goed dat ik tegen het eind van mijn afstuderen uit het raam staarde en dacht: ``Het zit er bijna op, hier op de universiteit\ppp''
Het treurige gevoel dat opkwam, gaf aan dat de omgeving, het onderwerp en het onderzoek leuk genoeg zijn om er nog een tijdje aan vast te knopen.
Ik vind het leuk om met de techniek bezig te zijn en als aio krijg je ruim de tijd om ergens lekker diep in te duiken.
En het schrijven van papers en een proefschrift?
Ach, het draait toch om de inhoud, dus dat zal wel loslopen\ppp
Ik ben mijn opa en Jan dankbaar voor de stimulans om eraan te beginnen.
Daarnaast heeft Gerard een goede plek geboden, zodat ik kon uitpluizen wat ik leuk vind.

Zoals iedere aio wel weet, kost schrijven toch iets meer tijd dan die \SI[output-decimal-marker={,}]{.25}{\percent}.
Echter, de scheidslijn tussen onderzoek doen en schrijven is heel dun; door de idee\"en op papier uit te werken, krijgen ze vorm en worden ze scherper.
En Marco, die al vroeg in het traject als begeleider betrokken raakte, heeft hier een zeer waardevolle bijdrage aan geleverd met zijn kennis, kritiek en uitdagingen.

Hoewel uiteindelijk alleen mijn naam op dit proefschrift staat, zijn er veel mensen die direct of indirect invloed hebben gehad op dit resultaat.
In willekeurige volgorde bedank ik enkelen van deze:
mijn kamergenoten Marco, Arjan, Robert en Koen voor de nodige discussies, overdenkingen en afleiding tijdens die stressvolle proefschriftschrijfdagen;
Berend die zich moedig durfde te wagen in de donkere hoekjes\footnote{Ho eens even, zoveel van die hoekjes zijn er helemaal niet!} van \Starburst;
Bert, die het weer moest ontgelden wanneer er een \mbox{(\noac{NFS}-)}server haperde, ook al kon hij daar niets aan doen;
Marlous, Thelma en Nicole voor de ondersteuning bij allerlei praktische zaken, zoals het boeken van snoepreisjes naar conferenties;
Pascal en Philip voor de geboden doorgroeimogelijkheden en een basis van een \LaTeX-template voor dit proefschrift, die via een ingewikkelde reeks van afgeleiden door (onder anderen?\@) Albert, Vincent, Maurice en Timon bij mij is gekomen, waar ook ik vervolgens hier en daar een 
\mbox{package\hspace{-.8em}\stick{1.1ex}{15}{t}\stick{1.3ex}{-10}{e}\stick{.9ex}{-20}{g}\stick{.2ex}{5}{e}naan}
heb geknutseld;
Hermen, die het is gelukd om de allerlaatste typefout uit m'n proefschrift te halen;
Christiaan, Mark en alle anderen van \noac{CAES} voor de pauzes en borrels, die altijd weer met een hoop lol en onzin werden gevuld.

De vier jaar (plus een beetje) zijn omgevlogen.
Ik heb genoten van wat ik heb gezien, geleerd en gedaan.
Mijn ouders hebben mij altijd gesteund en hebben meegeleefd tijdens verre reizen, waar ik ze erg dankbaar voor ben, maar ze kunnen het toch maar lastig volgen wat ik nu precies allemaal heb gedaan, ondanks mijn verwoede pogingen om het uit te leggen.
Wellicht dat dit proefschrift helderheid biedt, want eindelijk staat het nu eens allemaal netjes bij elkaar\ppp

Voor de laatste loodjes krijg ik hulp van mijn paranimfen, mijn beste vriend en geregelde lunchwandelgenoot Martijn, en aanstaande schoonvader Henk.
Tot slot, ik ben heel blij met mijn
\tikz[
	baseline=0,
	ear/.style={
		fill,shape=ellipse,black,anchor=center,inner sep=0,outer sep=0,
		minimum width=1ex,minimum height=.8ex,
		scale=.75,
	},
]{
	\node[inner sep=0,outer sep=0,anchor=base] at (0,0) {g};
	\node[overlay,ear,rotate=30] at (-.47ex,1.15ex) {};
	\node[overlay,ear,rotate=-30] at (.42ex,1.15ex) {};
}eliefde, Marjan,
die het altijd weer lukt om me op te vrolijken, wanneer er bijvoorbeeld een paper werd afgewezen, en die ik vaker dan eens heb beloofd om nu eens eerder thuis te komen, zodat we om een fatsoenlijke tijd kunnen eten.

\vspace{2em}

Jochem\\
Enschede, april 2014

